%
\documentclass[
12pt, % The default document font size, options: 10pt, 11pt, 12pt
%oneside, % Two side (alternating margins) for binding by default, uncomment to switch to one side
english, % ngerman for German
onehalfspacing, % singlespacing Single line spacing, alternatives: onehalfspacing or doublespacing
%draft, % Uncomment to enable draft mode (no pictures, no links, overfull hboxes indicated)
%nolistspacing, % If the document is onehalfspacing or doublespacing, uncomment this to set spacing in lists to single
liststotoc, % Uncomment to add the list of figures/tables/etc to the table of contents
toctotoc, % Uncomment to add the main table of contents to the table of contents
%parskip, % Uncomment to add space between paragraphs
nohyperref, % Uncomment to not load the hyperref package
headsepline, % Uncomment to get a line under the header
%chapterinoneline, % Uncomment to place the chapter title next to the number on one line
%consistentlayout, % Uncomment to change the layout of the declaration, abstract and acknowledgements pages to match the default layout
]{MastersDoctoralThesis} % The class file specifying the document structure

\usepackage[utf8]{inputenc} % Required for inputting international characters
\usepackage[T1]{fontenc} % Output font encoding for international characters

\usepackage{mathpazo} % Use the Palatino font by default

\usepackage[backend=bibtex,style=numeric,natbib=true, sorting=none]{biblatex} % Use the bibtex backend with the authoryear citation style (which resembles APA)

\addbibresource[label=refs]{example.bib} % The filename of the bibliography
\addbibresource[label=ownpubs]{ownpubs.bib} % The filename of the bibliography


\usepackage[autostyle=true]{csquotes} % Required to generate language-dependent quotes in the bibliography

\usepackage{arabxetex}

\usepackage[printonlyused,withpage]{acronym}

\usepackage{algorithmic}

\usepackage{adjustbox}


\usepackage{caption}
\usepackage{subcaption}
% \usepackage{subfig}

%-----------------
%	MARGIN SETTINGS
%-----------------

\geometry{
	paper=a4paper, % Change to letterpaper for US letter
	inner=2.5cm, % Inner margin
	outer=3.8cm, % Outer margin
	bindingoffset=.5cm, % Binding offset
	top=1.5cm, % Top margin
	bottom=1.5cm, % Bottom margin
	%showframe, % Uncomment to show how the type block is set on the page
}

%-----------------
%	THESIS INFORMATION
%-----------------

\thesistitle{Automated detection of COVID-19 cases in digital medical images using deep learning approaches} % Your thesis title, this is used in the title and abstract, print it elsewhere with \ttitle
\supervisor{} % Your supervisor's name, this is used in the title page, print it elsewhere with \supname
\examiner{} % Your examiner's name, this is not currently used anywhere in the template, print it elsewhere with \examname
\degree{Master of Science} % Your degree name, this is used in the title page and abstract, print it elsewhere with \degreename
\author{Mahmoud Z. Fetoh} % Your name, this is used in the title page and abstract, print it elsewhere with \authorname
\addresses{} % Your address, this is not currently used anywhere in the template, print it elsewhere with \addressname

\subject{Information Technology} % Your subject area, this is not currently used anywhere in the template, print it elsewhere with \subjectname
\keywords{ } % Keywords for your thesis, this is not currently used anywhere in the template, print it elsewhere with \keywordnames
\university{Menofia University} % Your university's name and URL, this is used in the title page and abstract, print it elsewhere with \univname
\department{Information Technology} % Your department's name and URL, this is used in the title page and abstract, print it elsewhere with \deptname
\group{  } % Your research group's name and URL, this is used in the title page, print it elsewhere with \groupname
\faculty{Faculty of computers and information} % Your faculty's name and URL, this is used in the title page and abstract, print it elsewhere with \facname

% \AtBeginDocument{
% \hypersetup{pdftitle=\ttitle} % Set the PDF's title to your title
% \hypersetup{pdfauthor=\authorname} % Set the PDF's author to your name
% \hypersetup{pdfkeywords=\keywordnames} % Set the PDF's keywords to your keywords
% }

\begin{document}


\frontmatter % Use roman page numbering style (i, ii, iii, iv...) for the pre-content pages

\pagestyle{plain} % Default to the plain heading style until the thesis style is called for the body content

%-----------------
%	TITLE PAGE
%-----------------

% \begin{titlepage}
% \begin{center}

% \includegraphics[width=2cm,height=2cm]{Figures/Unilogo.png}\\ % University/department logo - uncomment to place it

% \vspace*{.02\textheight}
% {\scshape\LARGE \univname\par}\vspace{0.2cm} % University name
% \textsc{\Large Faculty of Computers and Information}\\[0.5cm] % Thesis type

% \HRule \\[0.4cm] % Horizontal line
% {\huge \bfseries \ttitle\par}\vspace{0.4cm} % Thesis title
% \HRule \\[1.5cm] % Horizontal line
 
% \begin{minipage}[t]{0.4\textwidth}
% \begin{flushleft} \large
% \emph{Author:}\\
% \authorname % Author name - remove the \href bracket to remove the link
% \end{flushleft}
% \end{minipage}
% \begin{minipage}[t]{0.4\textwidth}
% \begin{flushright} \large
% \emph{Supervisors:} \\
% Prof. Khalid M. Amin
% Dr. Ahmed M. Hamad
% % \supname % Supervisor name - remove the \href bracket to remove the link  
% \end{flushright}
% \end{minipage}\\[2cm]
 
% % \vfill

% \large \textit{A thesis submitted in fulfillment of the requirements\\ for the degree of \degreename}\\[0.3cm] % University requirement text
% \textit{in the}\\[0.4cm]
% % \groupname\\\deptname\\[2cm] % Research group name and department name
% \deptname\\[2cm] %
% % \vfill

% {\large \today}\\[2cm] % Date
 
% \vfill
% \end{center}
% \end{titlepage}

%------------------
%	DECLARATION PAGE
%--------------------

% \begin{declaration}
% \addchaptertocentry{\authorshipname} % Add the declaration to the table of contents
% \noindent I, \authorname, declare that this thesis titled, \enquote{\ttitle} and the work presented in it are my own. I confirm that:

% \begin{itemize} 
% \item This work was done wholly or mainly while in candidature for a research degree at this University.
% \item Where any part of this thesis has previously been submitted for a degree or any other qualification at this University or any other institution, this has been clearly stated.
% \item Where I have consulted the published work of others, this is always clearly attributed.
% \item Where I have quoted from the work of others, the source is always given. With the exception of such quotations, this thesis is entirely my own work.
% \item I have acknowledged all main sources of help.
% \item Where the thesis is based on work done by myself jointly with others, I have made clear exactly what was done by others and what I have contributed myself.\\
% \end{itemize}
 
% \noindent Signed:\\
% \rule[0.5em]{25em}{0.5pt} % This prints a line for the signature
 
% \noindent Date:\\
% \rule[0.5em]{25em}{0.5pt} % This prints a line to write the date
% \end{declaration}

% \cleardoublepage

%-----------------
%	QUOTATION PAGE
%-----------------

% \vspace*{0.2\textheight}

% \noindent\enquote{\itshape Thanks to my solid academic training, today I can write hundreds of words on virtually any topic without possessing a shred of information, which is how I got a good job in journalism.}\bigbreak

% \hfill Dave Barry

%-----------------
%	ABSTRACT PAGE
%-----------------
\begin{abstract}
\addchaptertocentry{\abstractname} % Add the abstract to the table of contents
COVID-19 is a severe respiratory tract infection. COVID-19 caused by SARS-CoV-2 can readily spread through contact with an infected person. Monotonically increasing SARS-CoV-2 infections have not only wasted lives but also severely damaged the financial systems of both developing and developed countries. This high spread rate pressure on the health care systems which raises the need for fast methods for diagnosing this disease. Convolutional Neural Networks (CNN) show great success for various computer vision tasks. However, CNN is a scale-variant model and is computationally expensive. In this Thesis, novel architectures are proposed for multiscale feature extraction and classification and lightweight architecture for COVID-19 diagnosing. The proposed I which is a lightweight CNN model exploits spatial kernel separability to reduce the number of the training parameters to a large extent and regularize the model to only learn linear kernel. Furthermore, This model uses residual connection and batch normalization extensively to maintain the network stability during the training process and provide the model with the regularization effect to reduce overfitting. Proposed CNN II learns multiscale features using a pyramid of shared convolution kernels with different atrous rates. This scale-invariant CNN uses an attention-based mechanism that is used to guide and select the correct scale for each input. Proposed CNN II is an end-to-end trainable network and exploits a novel augmentation technique, Texture Augmentation, to reduce overfitting. The lightweight architecture is trained using the QaTa-Cov19 benchmark dataset achieving  100\% for accuracy, sensitivity, precision, and F1-score with a very low parameter count (150K) compared with the other methods in the literature.  Proposed method II achieved a 0.9929 for $F1-score$ tested on the QaTa-Cov19 benchmark dataset with a total of $5,040,571$ trainable parameters.
\end{abstract}
	
%-----------------
%	ACKNOWLEDGEMENTS
%-----------------

\begin{acknowledgements}
\addchaptertocentry{\acknowledgementname} % Add the acknowledgements to the table of contents

\indent \textbf{Praise be to ALLAH first and last. Prayers and peace be upon MOHAMMED the Messenger of ALLAH}.\\

I would like to thank my supervisors and my teachers \textit{Dr. Ahmed M. Hamad} and \textit{Prof. Khalid M. Amin} for their help and guide during my Study. They have been helpful with background information and have continually encouraged me and helped me with comments on my work. They always had time to discuss new ideas and give feedback on early ideas and research problems.
\end{acknowledgements}

%-----------------
%	LIST OF CONTENTS/FIGURES/TABLES PAGES
%-----------------

\tableofcontents % Prints the main table of contents

\listoffigures % Prints the list of figures

\listoftables % Prints the list of tables

% \listofsymbles

%-----------------
%	ABBREVIATIONS
%-----------------

\begin{abbreviations}{ll} % Include a list of abbreviations (a table of two columns)

COVID-19 & COronaVIrus Disease 2019\\
SARS-CoV-2 & severe acute respiratory syndrome coronavirus 2\\
WHO &  World Health Organization\\
rRT-PCR & real-time reverse transcription polymerase chain reaction \\
TMA &  transcription-mediated amplification\\
RT-LAMP & reverse transcription loop-mediated isothermal amplification\\
CXR  & Chest X-rays \\
CNN  & convolutional Neural Network\\
tanh & Hyperbolic tangent \\
ReLU & Rectified Linear Unit \\
BoW & Bag-of-Words \\
SVM  & Support Vector Machine \\
ICS & Internal Covariate Shift\\
BN & Batch Normalization \\
ASPP & Atrous Spatial pyramid \\
RC &Representation-based classification \\
BGWO  & binary grey wolf optimization \\
ANN  & Artificial neural network \\
PCA & principal component analysis \\
DT  & Decision Tree \\
NB  &  Naive Bayes \\
FOSF &  first order statistical features  \\
GLCM & Gray Level Co-occurrence Matrix \\
HOG &  Histogram of Oriented Gradients\\
KNN  & k-nearest neighbors \\
RBF  & radial basis function \\
RSB & Residual Separated Block\\
SWASPP & Spatially weighted Atrous Spatial Pyramid\\
DSWASPP & Densely stacked SWASPP\\
SPP &  Spatial Pyramid Pooling\\
SASPP-net & Switchable Atrous Spatial Pyramid Pooling\\
FX & feature extraction modules\\
Adam  & Adaptive Moment Estimation \\

% \textbf{LAH} & \textbf{L}ist \textbf{A}bbreviations \textbf{H}ere\\
% \textbf{WSF} & \textbf{W}hat (it) \textbf{S}tands \textbf{F}or\\
\end{abbreviations}
% \begin{acronym}
% 	\acro{USA}{Uited States of America}
% \end{acronym} 

%-----------------
%	PHYSICAL CONSTANTS/OTHER DEFINITIONS
%-----------------

% \begin{constants}{lr@{${}={}$}l} % The list of physical constants is a three column table

% % The \SI{}{} command is provided by the siunitx package, see its documentation for instructions on how to use it

% Speed of Light & $c_{0}$ & \SI{2.99792458e8}{\meter\per\second} (exact)\\
% %Constant Name & $Symbol$ & $Constant Value$ with units\\

% \end{constants}

%-----------------
%	SYMBOLS
%-----------------

\begin{symbols}{lll} % Include a list of Symbols (a three column table)

$T$ & Softmax temperature determine the steepness of the function & \si{\meter} \\
$P$ & power & \si{\watt} (\si{\joule\per\second}) \\
%Symbol & Name & Unit \\

\addlinespace % Gap to separate the Roman symbols from the Greek

$\omega$ & angular frequency & \si{\radian} \\

\end{symbols}

%-----------------
%	DEDICATION
%-----------------

\dedicatory{To my Family} 



%-----------------
%	THESIS CONTENT - CHAPTERS
%--------------------

\mainmatter % Begin numeric (1,2,3...) page numbering

\pagestyle{thesis} % Return the page headers back to the "thesis" style

   
% Include the chapters of the thesis as separate files from the Chapters folder
% Uncomment the lines as you write the chapters

% Chapter Template

\chapter{Introduction} % Main chapter title

% Importance
% Motivation
% Application
% challenges
% thises Content
% 3Pages of intro 
% RelatedW0Rk  



\label{chp:intro} % Change X to a consecutive number; for referencing this chapter elsewhere, use \ref{ChapterX}
\section{Novel Coronavirus Disease}

COronaVIrus Disease 2019 (COVID-19) is a severe respiratory tract infection that can range from mild to lethal \cite{acter2020evolution}. COVID-19 is a contagious disease that can readily spread through direct or indirect contact with an infected person \cite{singhal2020review}. COVID-19 is caused by severe acute respiratory syndrome coronavirus 2 (SARS-CoV-2). COVID-19 initially appeared in Wuhan, China late 2019 and spread worldwide \cite{hui2020continuing}. World Health Organization (WHO) declared the outbreak of COVID-19 as a Public Health Emergency of International Concern in January 2020, and a pandemic in March 2020 \cite{platto2020covid19}.
Ongoing SARS-CoV-2 infections have not only devastated human lives but also significantly damaged the financial health of both developing and developed countries. Therefore, there is an urgent need to control the pandemic by accelerating the development and mass production of efficacious vaccines against SARS-CoV-2. Healthcare practitioners, researchers, and policymakers around the globe were thrown a challenge to deliver adequate prevention and treatment modalities to combat the pandemic. From the initial stage of this pandemic, scientists were focused on either repurposing the existing drugs or developing vaccines against COVID-19 \cite{le2020evolution}. Figure \ref{NorCovCXR} illustrates the difference between the COVID-19 and normal lungs.

Typical diagnostic tools for COVID-19 are virus’ nucleic acid by real-time reverse transcription polymerase chain reaction (rRT-PCR), transcription-mediated amplification (TMA), or reverse transcription loop-mediated isothermal amplification (RT-LAMP) from a nasopharyngeal swab \cite{tahamtan2020real}.  These manual traditional methods are time-consuming and complex. Chest X-rays (CXR) and chest CT offer fast screening methods for COVID-19 \cite{salehi2020coronavirus}\cite{wu2020new}\cite{zu2020coronavirus}. CXR and chest CT are preferred when RT-PCR testing is not available in time \cite{erickson1993advanced}. CXR has many advantages over chest CT including the widespread of acquisition devices, low cost, and the speed of the acquisition\cite{narin2021automatic}\cite{brenner2007computed}\cite{rubin2020role}\cite{shi2020review}. These advantages lead CXR to be a fixed routine for hindering COVID-19 spread. 

\begin{figure}%
    \centering
    \begin{subfigure}[b]{0.4\textwidth}
        \centering
        \includegraphics[width=\textwidth]{Figures/introNormCXR.png}
        \caption{Chest X-ray image of normal lung.}
        \label{NorCXR}
    \end{subfigure}
    % \hfill
     \begin{subfigure}[b]{0.4\textwidth}
         \centering
         \includegraphics[width=\textwidth]{Figures/intoCovidCXR.png}
         \caption{Chest X-Ray image of COVID-19 patient}
         \label{CovCXR}
     \end{subfigure}
    \caption{Samples of CXR images of Normal and COVID-19 pneumonia}%
    \label{NorCovCXR}%
\end{figure}
\section{Research Problem Statement} 
The rapid spread of COVID-19 caused by SARS-CoV-2 has not only resulted in significant loss of life but has also strained the financial systems of both developing and developed countries. This widespread transmission places immense pressure on healthcare systems, emphasizing the urgent need for efficient diagnostic methods. While CNNs have shown promise in various computer vision tasks, they have limitations such as scalability and computational cost.
This thesis tackles the following problems of the COVID-19 classification.
\begin{itemize}
    \item High classification accuracy for COVID-19 detection.
    \item Proposing a model with lower computational requirements.
\end{itemize} 
 This thesis aims to address these challenges by proposing two novel CNN architectures for COVID-19 diagnosis: one focusing on lightweight design and the other on multiscale feature extraction. The primary problem addressed in this research is to develop effective and efficient CNN models for COVID-19 diagnosis that can provide high accuracy while minimizing computational complexity and overfitting. These models are evaluated using the QaTa-Cov19 benchmark dataset, aiming to achieve superior performance with significantly fewer trainable parameters compared to existing methods in the literature.

\section{Motivation}
Figure \ref{hospitalRoutine} illustrates a typical Egyptian hospital routine for reducing COVID-19 spread among healthcare workers. This process is performed for every visitor to the hospital. Phases of CBC tests and PCR tests are automated and do not consume time. While CXR image diagnosing involves human factors that act as a bottleneck of the process. This process can be fully automated using image classification models. As a consequence \textit{1)} Total time required for every visitor will be reduced. \textit{2)} The load of radiologists is reduced. 

\begin{figure}%
    \centering
        \includegraphics[width=\textwidth]{Figures/HosPitalCovidRoutine.png}
        \caption{Typical hospitals routine of reducing COVID-19 spread}
        \label{hospitalRoutine}
\end{figure}

Deep learning \cite{lecun2015deep} models, namely convolutional Neural Network (CNN) \cite{lecun1989handwritten}, have shown a great performance in computer vision problems such as object detection \cite{erhan2014scalable}\cite{girshick2014rich}\cite{sermanet2013overfeat}\cite{redmon2016you} and object recognition \cite{simonyan2014very}\cite{he2016deep}. CNN is initially introduced in \cite{lecun1989handwritten}. CNN  is based on hierarchical learning of the convolutional kernels which are organized in layers \cite{krizhevsky2012imagenet}. The function of lower-order layers is to learn low-level features such as edges and corner points \cite{zeiler2014visualizing}. Higher order layers learn high-level features such as objects \cite{zeiler2014visualizing}. Typical CNN architecture is several convolutional layers that are connected sequentially \cite{simonyan2014very}. This sequential connection of the layers does not scale well for deep CNN \cite{he2016deep}. CNN shows great performance when it has a large number of layers \cite{he2016deep}. To allow CNN to take advantage of the deep architecture, residual connections are used \cite{he2016deep}. Residual connections solve the vanishing gradient problem when gradients approach zeros. Also, residual connections allow the reuse of earlier features \cite{huang2017densely}. Layers with residual connections are easier to optimize than plain layers as they can easily approximate the identity mapping \cite{he2016deep}. In this thesis, CNN is used to automate and accelerate the diagnosing pipeline of COVID-19. Figure \ref{solMeth} represents proposed methodologies.


\begin{figure}%
    \centering
        \includegraphics[width=\textwidth]{Figures/SolutionMethodologies.png}
        \caption{Solution Methodologies proposed in this thesis}
        \label{solMeth}
\end{figure}


\section{Detection Challenges}
Like many computer vision problems, COVID-19 has the following challenges
\begin{itemize}

\item \textbf{Availability of Data.} CNN is a deep learning technique that requires a large number of training Images. However, the current COVID-19 dataset are small-size dataset which is challenging to train large networks with. 
\item \textbf{Scale.} Like many computer vision models CNN is a scale variant. CNN can not recognize images with scales different from the scales it trained on.
\item \textbf{Vanishing Gradient Problem.} Deep CNN suffers a vanishing gradient which prevents CNN parameters from being updated.
\item \textbf{Exploding Gradient Problem.} Deep CNN suffers Exploding gradient which makes CNN diverge.
\end{itemize}
Proposed Systems try to overcome these problems as will be detailed in chapters \ref{chp:proposed1} and \ref{chp:proposed2}.

\section{Thesis Objective and Contributions}
The objective of this thesis is to improve classification accuracy and reduce the computational complexity of detecting COVID-19 cases from CXR Images. Contributions of this thesis are as follows.
\begin{itemize}
    \item \textit{Lightweight classification model.} A Lightweight classification model with a very low parameter number is proposed for the classification of the CXR images. It reduces the computational complexity which allows deployment on mobile devices and saves bandwidth of the network during the model distribution process. 
    \item \textit{Scale invariant model.}  Pneumonia scales in CXR are not uniformly distributed which prevents CNN from recognizing infrequent scales. Scale-invariant model is proposed for the detection of COVID-19 pneumonia at different scales.
    \item \textit{High Detection accuracy.} Proposed systems provide superior performance according to various classification metrics.
\end{itemize}

\section{Thesis Organization}
This thesis is organized as follows: 
\begin{itemize}
    \item \textbf{Chapter \ref{chp:background}}: includes required Background to understand the Thesis.Also includes and illustrates the recent and related work in the COVID-19 detection literature. 
    \item \textbf{Chapter \ref{chp:proposed1}}: presents the proposed work I which presents a lightweight classification model. Also presents the proposed work II which includes the scale-invariant model for COVID-19 classification. 
    \item \textbf{Chapter \ref{chp:results}}: illustrates the experimental results for both proposed work I and II and quantitative analysis of the proposed work I and II is provided.
    \item \textbf{Chapter \ref{chp:concl}}: concludes the thesis and provides planning for the future work as an extension of the proposed approach
\end{itemize}
% Chapter Template

\chapter{Background} % Main chapter title

\label{chp:background} % Change X to a consecutive number; for referencing this chapter elsewhere, use \ref{ChapterX}

This chapter includes required background to understand the thesis proposal presented in the following chapters.

\section{Convolutional Neural Networks (CNN)}
CNN is initially introduced by LeCun \cite{lecun1989handwritten} which is based on learning adaptive convolutional kernels. CNN consist of two parts convbase and densebase parts. For the Convbase instead of connecting all of the units in a layer to all the units in a preceding layer, convolutional networks organize each layer into feature maps \cite{lecun1989handwritten}, which
can be though of as parallel planes or channels. In a convolutional layer, the weighted sums are only performed within a small local window \textit{i.e)} receptive field, and weights are identical for all pixels, just as in regular shift-invariant image convolution and correlation. This parameter sharing reduce the required total number of parameter and allows learning shift invariant convolutional kernels. These convolutional kernels produce equivariant features maps. Fig. \ref{lenet} represents typical CNN architecture of LeNet.
\begin{figure}
    \begin{center}
        \includegraphics[width=\textwidth]{Figures/LeNetCNN.jpeg}
        \caption{Typical CNN architecture}
        \label{lenet}
    \end{center}
\end{figure}
\subsection{Convolutional Layer}
The building block of the convolutional layer is the 2D convolutional kernel. Fig. \ref{conv2Dlayer} illustrates the 2D convolutional layer. Each 2D convolution kernel takes as input all of the $C_{i-1}$ channels in the preceding layer, windowed to a small area, and produces the values in one of the $C_{i}$ channels in the next layer. For each of the output channels, we have $K^2\times C_{i-1}$ kernel weights, so the total number of learnable parameters in each convolutional layer is $K^2\times C_{i-1} \times C_{i}$. In Fig. \ref{conv2Dlayer}, we have $C_{i-1} = 6$ input channels and $C_{i} = 4$ output channels, with an $K = 3$ convolution window, for a total of $9 \times 6 \times 4$ learnable weights, shown in the middle column of the figure. Since the convolution is applied at each of the $W \times H$ pixels in a given layer, the amount of computation (multiply-adds) in each forward and backward pass
over one sample in a given layer is $W\times H \times K^{2} \times C_{i-1} \times C_{i}$.
To fully determine the behavior of a convolutional layer, we still need to specify the following hyperparameter:
\begin{itemize}
    \item \textbf{Padding.} Padding is used to preserve the spatial dimension of the input feature map after the convolution operation is performed. Typically, it is performed by inserting $\lfloor K/2 \rfloor$ columns for both sides and $\lfloor K/2 \rfloor$ rows to the top and bottom.
    \item \textbf{Stride.} Stride is the step taken between two centers when performing the convolution operation. Typically, Stride is equal to $1$. Stride can act as down sampling operation that can be performed instead of the pooling operation.
    \item Dilation.
    \item Grouping.
\end{itemize}





\begin{figure}
    \begin{center}
        \includegraphics[width=\textwidth]{Figures/2DConvKernels.png}
        \caption{Convolutional layer with single input feature map and four convolutional kernels}
        \label{conv2Dlayer}
    \end{center}
\end{figure}


Batch normalization algorithm is described as follows:

\vspace*{1.3\baselineskip}
\begin{algorithmic}[1]

\REQUIRE : Minibatch activation values $x$ : $\mathcal B = \{x_{1,\ldots,m}\}$ ; parameters to be learned $\gamma$ ,$\beta$.

\ENSURE  : $\{y_i = \mathrm{BN}_{\gamma,\beta}(x_i)\}$
\vspace*{.7\baselineskip}
\STATE $\mu_{\mathcal B} \leftarrow \frac1m \sum_{i = 1}^m x_i$
\vspace*{.7\baselineskip}
\STATE $\sigma^2_{\mathcal B} \leftarrow \frac1m \sum_{i=1}^m (x_i - \mu_{\mathcal B})^2$
\vspace*{.7\baselineskip}
\STATE $\hat x_i \leftarrow \frac{x_i - \mu_{\mathcal B}}{\sqrt{\sigma_{\mathcal B}^2 + \epsilon}}$
\vspace*{.7\baselineskip}
\STATE $y_i \leftarrow \gamma \hat x_i + \beta \equiv \mathrm{BN}_{\gamma,\beta}(x_i)$

\end{algorithmic} 
% Chapter Template

\chapter{Literature Review} % Main chapter title

\label{ChapterX} % Change X to a consecutive number; for referencing this chapter elsewhere, use \ref{ChapterX}

% Chapter Template

\chapter{Proposed Work} % Main chapter title

\label{chp:proposed} % Change X to a consecutive number; for referencing this chapter elsewhere, use \ref{ChapterX}
In this chapter two methodologies for COVID19 detection is presented. Methodology I present light CNN architecture for COVID19 detection which does not require large computational power to run. Methodology I does not care about the scale of the pneumonia. While methodology II does.


\begin{figure}[th]
    \centering
    \includegraphics[height=40mm,width=8.0cm]{Figures/fig1.jpg}
    % \decoRule
    \caption{The phases of the proposed method I.}
    \label{fig1}
    \end{figure}

\section{Methodology I}
In this section, a  proposed  method I to detect COVID-19 disease from chest X-Ray images is presented. The proposed method exploits CNN model to classify the input chest X-Ray image to one of two categories; normal case or Covid-19 case. The proposed  method I consists of three phases: preprocessing, feature extraction, and classification. The proposed method phases are shown in Fig.\Ref{fig1}. 

\subsection{Preprocessing Phase}

The preprocessing phase is responsible for resizing and normalizing the  input  chest X-Ray images. The pre-processing phase is employed to maintain the numerical stability of the model and reduce the co-variance shift \cite{lecun1989handwritten}. In addition, this phase leads the learning model of CNN model to reduce  the required overhead to adapt to the different scales of different features of the input data. Reshaping size is determined empirically. The input  chest X-Ray image is re-sized and then  adapted and normalized to a normal distribution as follows:

\begin{equation}
Y := \frac{x_i - \mu_{\mathcal B}}{\sqrt{\sigma_{\mathcal B}^2 + \epsilon}}
\label{eq1}
\end{equation}
where $\mu$ and $\sigma$ is the mean  and standard deviation of chest X-Ray image (X), respectively.

After re-sizing the input chest X-Ray image, the input image is normalized to have a zero mean and unit standard deviation. Then,  the image can be scaled and shifted with a normalization parameter which is determined and adapted by the training dataset during the training process according to the following equation: 

\begin{equation}
Z := w_1 Y + w_2
\label{eq2}
\end{equation}
where $w_1$ and $w_2$ are a trainable parameter.

Unlike the  normalization method presented in \cite{ioffe2015batch}, the batch normalization process presented in this paper has $z$-score normalization parameter that is used in both training and validation phases.


\subsection{Feature Extraction and Classification}

CNN models achieved an outstanding success in image recognition \cite{lecun2015deep}. This phase  is responsible for extracting spatial features from the normalized chest X-Ray image using a tailored CNN model.  This phase is based on learning the CNN model by the input preprocessed chest X-Ray images. The design of the tailored CNN model is described as follows: 

\textit{1) Separable CNN kernels}
\begin{figure*}
\begin{center}
\includegraphics[height=33mm,width=14.0cm]{Figures/fig2.jpg}
\caption{Separable convolution  $Gy$ and $Gx$ have kernel size of $M\times1$ and $1 \times M$. The combination of these kernels is approximately a $M\times M$ kernel  and depth wise convolution are applied by a $1\times1$ convolution. The output depth  is padded with zeros to have the same spatial size of  $Gy, Gx$. $Gy, Gx$ are performed channel wise. }
\label{fig2}
\end{center}

\end{figure*}


\begin{figure}
\begin{center}
\includegraphics[height=34mm,width=7.0cm]{Figures/fig3.jpg}
\caption{Separated Convolutional Layer}{ composed of three consecutive layers. The first Convolutional layer has a kernel size of $(M\times1)$ and $D$  convolutional neuron. The second layer  operates in the same way as the first layer but it has a kernel of size $(1\times M)$ and $D$ convolutional neuron. The third layer is the convolutional layer with kernel of size $(1\times1)$ and number of convolutional neuron is $N$.}
\label{fig3}
\end{center}

\end{figure}
    
Kernel separability\cite{rigamonti2013learning} \cite{szegedy2017inception} is based on decomposing a 2D convolution kernel to linear combinations of two 1D vectors which leads to a large reduction in  the total number of resulting parameters. For example, a 2D kernel of size $9 \times 9$ has a total number of $9^2 = 81$  trained parameters. Whereas in the case of separating this 2D kernel to  linear combinations of two 1D vectors of sizes $9 \times 1$ and $1 \times 9$, this results in a total number of  $9 + 9 = 18$ trained parameters. As a consequence, kernel separability reduces the number of CNN model operations (such as the multiplication and the addition). A  2D kernel of $k \times k$ applied for 2D signal with spatial dimensions of $ M \times N$ has a total number of  $(N-4)(M-4)\times k^2$ operations but in case of  applying kernel separability  yields $2(N-4)(M-4)k$ operations. The flow of separated convolution operations are summarized in Fig. \ref{fig2}. Fig. \ref{fig3} represents the structure, denoted by Separated Convolutional Layer, used in the proposed method with kernel size of $(M\times N)$ and satisfying the convolutional kernel separability. Separated Convolutional Layer is composed of three consecutive layers. The first convolutional layer has a kernel size of $(M\times1)$ and the number of convolutional neuron and  filters are equal to the number of channels as the input feature map and the convolution operations are performed in a channel wise. The second layer  operates in the same way as the first layer but it has a kernel of size $(1\times M)$. The third layer is the convolutional layer with kernel of size $(1\times1)$ and number of convolutional neuron is $N$. The collaboration of the three layers are  connected to preform similarly to the convolutional layer with kernel size of $(M\times M)$ and number of neuron and filter are the same as $N$ but with large difference in the performance.



\textit{2)  Batch Normalization and  Activation function}

In the proposed method linear separable convolutional kernels are followed by a batch normalization and an activation function. Rectified Linear Unit (ReLU) \cite{he2015delving} is a nonlinear activation that allows the network to fit and approximate highly non-linear datasets distribution. The proposed method employs the batch normalization which is described in \cite{ioffe2015batch}. 

Batch Normalization \cite{ioffe2015batch} reduces internal covariate shift produced as a result of  moving between layers during the feedforward procedure \cite{ioffe2015batch}. Batch Normalization makes the loss landscape smoother and reduces the number of saddle points \cite{santurkar2018does} which allows to use higher learning rates. Using a higher learning rate makes the network training  faster\cite{ioffe2015batch}. Batch normalization reduces the vanishing gradient problem and exploding gradient problem as it makes the resulted activation scale independent from the trainable parameter scale\cite{ioffe2015batch}. Batch normalization has the effect of regularization because of the inherited randomness when selecting the batch sample\cite{ioffe2015batch} which help the generalization to unseen chest X-Ray image.

\textit{3) Deep and larger receptive field Network design}

Deeper convolutional neural network design is a very important task for any image recognition task \cite{he2016deep}. Training a deeper network is very expensive and has many challenges such as vanishing gradient problem, exploding gradient problem, and degradation problem \cite{he2016deep}. Exploding gradient problem occurs  when the  gradient update becomes very large (approaching infinity) resulting in the network diversion. Vanishing gradient problem occurs when the  gradient update becomes very small (approaching zero) resulting in preventing the parameter update for early layers\cite{ioffe2015batch} and preventing the network to learn new patterns. Batch normalization \cite{ioffe2015batch} and the use of ReLU activation function \cite{krizhevsky2012imagenet} alleviate these two problems.

The deep layers of CNN networks sometimes need to  approximate the identity function which is not a simple task especially  with the existence of a non-linear functions. Residual connection\cite{he2016deep} overcomes this problem by using skip connection as shown in Fig. \ref{fig4}.
Fig. \ref{fig4} represents the building block layer of the feature extraction phase, denoted by stack of Residual Separated Block  (RSB). RSB consists of four layers of separated convolutional layers, each layer is followed by a batch normalization and an activation function. It has an output of depth $N$ where each sublayer produces an output of depth $N/4$ which is concatenated at the end of the layer to produce a depth  $N$. RSB produces a feature map that includes both low level features and high level features.

\begin{figure*}
\begin{center}
\includegraphics[height=38mm,width=14.0cm]{Figures/fig4.jpg}
\caption{The stack of residual separated block  (RSB) consists of four layer of separated convolutional layer each of which is followed by batch normalization and activation function.}
\label{fig4}
\end{center}
\end{figure*}

Unlike the traditional neural network, which is fully connected to the previous layer, convolutional neural network is connected locally to a local region of the previous feature map. This introduces the concept of the network receptive field \cite{luo2016understanding}. Receptive field should be large enough to capture large patterns in the input chest X-Ray image. Therefore, any consecutive convolutional layers in the proposed method without a pooling layer in between a larger kernel size is used in one of them. Residual Separated block, RSB, in Fig. \ref{fig4} may have kernel sizes of 3, 5, 7, and 9, respectively.\\
Fig. \ref{fig5} Represent a complete CNN architecture.

\begin{figure*}
\begin{center}
\includegraphics[height=37mm,width=14.0cm]{Figures/fig5.jpg}
\caption{The complete proposed tailored CNN architecture.}
\label{fig5}
\end{center}
\end{figure*}


 
% Chapter Template

\chapter{Proposed Methodology II} % Main chapter title

\label{chp:proposed2} % Change X to a consecutive number; for referencing this chapter elsewhere, use \ref{ChapterX}


CNN, like many computer vision models, is a scale-variant \cite{van2017learning} model such that it cannot recognize objects at various scales unless it explicitly trained to recognize such objects.  This chapter presents a CNN architecture that learn multiscale features using scale pyramid. Scale pyramid is constructed using atrous convolution. The correct scale from scale pyramid is selected using the spatial attention mechanism.
\section{Methodology II} 

\begin{center}
    \begin{figure*}[htbp]
    \centerline{\includegraphics[height=40mm,width=15cm]{Figures/ProposedPipe.png}}
    \caption{Proposed method for COVID-19 classification from CXR images.}\label{ProposedPipe}\end{figure*}\end{center}
    
The proposed system presented in this chapter proposes a novel CNN micro-architecture model for learning  scale-invariant features of   CXR images and then classifies these features into normal or COVID-19 cases. Fig. \ref{ProposedPipe} illustrates the proposed end-to-end pipeline  of the proposed system. The proposed system depends on a novel Spatially weighted Atrous Spatial Pyramid Pooling (SWASPP) to extract multi-scale features of input CXR images. A novel attention model is then used to fuse the extracted multi-scale  features and select the relevant scale features that the next CNN network should consider.
\subsection{Data augmentation}


\begin{center}
    \begin{figure}[htbp]
    \centerline{\includegraphics[height=30mm,width=9cm]{Figures/TexAug.PNG}}
    \caption{Texture Augmentation module}
    \label{texaug}
    \end{figure}
    \end{center}
The first phase of the proposed CXR classification system is data augmentation. Data augmentation is used to reduce the overfitting and artificially enlarge the training dataset \cite{krizhevsky2012imagenet}. The input CXR images are augmented  using texture augmentation.  Texture augmentation is performed by adding a multiplicative normally distributed noises to the frequency spectrum of the input image. Noise is modeled using $\mathcal{N}(\mu = 1,\,\sigma = 0.3)$. Fig. \ref{texaug} illustrates texture augmentation process. Fig. \ref{resltaug} shows the resultant CXR image. A standard augmentation such as random rotation, horizontal flipping, and vertical flipping are included in the augmentation process. 

\begin{center}
    \begin{figure}[htbp]
    \centerline{\includegraphics[height=40mm,width=9cm]{Figures/freqJitt.png}}
    \caption{Texture Augmentation}{The resulting CXR image from Texture augmentation \textbf{left}: is the original image. \textbf{Right} is the augmented  CXR Image}
    \label{resltaug}
    \end{figure}
    \end{center} 
    

\begin{center}
\begin{figure}[htbp]
\centerline{\includegraphics[height=50mm,width=9cm]{Figures/SWASPP.PNG}}
\caption{Spatially weighted atrous spatial Pyramid Pooling (SWASPP) interal layers within dashed square are parameter shared.}
\label{swaspp}
\end{figure}
\end{center}

\subsection{Spatially Weighted Atrous Spatial Pyramid Pooling}

Atrous convolution is a powerful technique for adjusting the resolution of  convolutional kernels. This allows to effectively enlarge the field-of-view of the kernel without increasing neither the number of kernel parameters  nor the computational complexity of  the convolution performance. A novel spatially weighted atrous spatial pyramid pooling (SWASPP) micro-architecture is presented. Fig. \ref{swaspp} shows the architecture structure. In Fig. \ref{swaspp}, internal layers, bounded by dashed square, are parameter-shared and have different atrous rates. These layers are responsible for extracting multi-scale features. Sharing of the parameters enforce these layers to learn scale-invariant features. For a given input CXR image,  three scales feature maps are produced. Each feature map  corresponds to a particular scale. 

\begin{center}
    \begin{figure}[htbp]
    \centerline{\includegraphics[height=60mm,width=3.5cm]{Figures/AttentionModUl.PNG}}
    \caption{Attantion module structure used by SWASPP micro-architecture}
    \label{attain}
    \end{figure}
    \end{center}
    
To fuse the produced feature maps representing different scales of the input image, an attention module is added. Attention module can be thought as a pixel level classification of which scale does this spatial position belong. Fig. \ref{attain} illustrates the proposed attention module structure. Proposed attention module generates four heatmaps. The first three heatmaps correspond to the three scale feature maps while the remaining heatmap corresponds to the input feature map itself. These heatmaps are summed  up to one (\textit{i.e.,} for a  spatial position 
$(x, y)$, $\sum_{i =1}^{4} H(i,x,y) = 1$ where $H(i,x,y)$ is the $i$ heatmap produced by the attention module). To make sure this property holds, softmax function is used. 

The proposed mirco-architecture uses a pixel level weights produced by corresponding attention module rather than a single weight value for each scale. A single input CXR image may have multiple COVID-19 pneumonia scales which effectively lead to simply averaging the scale space when using single weight for each scale on scale space. In SWASPP, every convolution operation is followed by a BN and leakyReLU \cite{krizhevsky2012imagenet} non-linearity except the re-projection layers that used to project back to the input space. 
BN allows the use of larger learning rate\cite{ioffe2015batch} and makes network stable during training\cite{ioffe2015batch}. BN makes the loss landscape of the optimization problem significantly smoother\cite{santurkar2018does}.
leakyReLU is used to reduce the vanishing gradient problem \cite{krizhevsky2012imagenet}.
A bottleneck is introduced within both the attention module and multi-scale feature extractor layer. A bottleneck in SWASPP is used to project the input feature map of dimension $C_{in}\times H\times W$ to $32\times H\times W$ then re-project back to $C_{in}\times H\times W$. Multi-scale feature extraction is preformed on the projected dimension. Same logic is applied to the attention module where the input feature map is projected to a dimension of $16\times H\times W$.
This bottleneck allows the efficient use of model capacity and reduce the network computational complexity \cite{huang2017densely}. 

\subsection{Proposed CNN Architecture}
SWASPP is densely stacked \cite{huang2017densely} together as Fig. \ref{denseB} illustrates. This kind of connectivity allows implicit deep supervisions as each layer is effectively connected to the last layer using shorter path also facilitate feature reuse \cite{huang2017densely}. Residual layers are easier to optimize if the required mapping is the identity mapping or simply near to it \cite{he2016deep}. Densely stacked SWASPP is denoted by (DSWASPP). Convolutional part of proposed model consists of stacking six DSWASPP layers such that the first four layers are interconnected using maxpooling to reduce the spatial size and enlarge the Network receptive field. A single level Spatial Pyramid Pooling (SPP) \cite{he2015spatial} is added after to produce a fixed size feature vector for a variable size input. SPP layer divides the input feature map into $10\times 10 = 100$ bins then performs a $max$ for each bin as an aggregation function. 
\begin{center}
\begin{figure}[htbp]
\centerline{\includegraphics[height=30mm,width=6cm]{Figures/DensResd.PNG}}
\caption{Densely connected SWASPP (DSWASPP): is a stack of densely connected SWASPP, such that the output of any SWASPP is Concatenated to the input of all next layers. All the three layers produce an output of dimension of $C_{in} \times H \times W$.}
\label{denseB}
\end{figure}
\end{center}
The fixed length feature vector produced by SPP is used as an input to dropout \cite{srivastava2014dropout} layer. Dropout layer randomly sets the activation of to $0$ with a probability of $0.5$. Dropout prevents the overfitting and reduce complex co-adaptation between the neurons allowing them to learn better representation \cite{srivastava2014dropout}. It allow implicit ensempling of exponential number of sampled thin network from the original network which enhance the network performance \cite{srivastava2014dropout}. The result of dropout layer is used as input to the classification network. Classification network consists of a fully connected layers with a $3$ Dense layers such that the output layer is 2-neuron for binary classification \textit{i.e)} COVID19 or not. Table \ref{PCNN} shows the details of the proposed architecture.

\renewcommand{\arraystretch}{1.5}
\begin{table}[htbp]
    \caption{Proposed CNN architecture of methodology II}
    \begin{center}
    \begin{tabular}{|c|c|c|c|}
    \hline
    \textbf{Layer}&\multicolumn{3}{|c|}{\textbf{Proposed CNN Architecture of Methodology II}} \\
    \cline{2-4} 
    \textbf{Name} & \textbf{\textit{Input Shape}}& \textbf{\textit{Output Shape}}& \textbf{\textit{Param. Count}} \\
    \hline
    Input layer & - & $1 \times 320 \times 320$ & 0 \\
    \hline
    BatchNorm-1 & $1 \times 320 \times 320$ & $1 \times 320 \times 320$ & 2 \\
    \hline
    DSWASPP-1& $1 \times 320 \times 320$ & $32 \times 320 \times 320$ & 121,035  \\
    \hline
    Maxpooling-1& $32 \times 320 \times 320$ &$32 \times 160 \times 160$ & 0 \\
    \hline
    DSWASPP-2& $32 \times 160 \times 160$ & $64 \times 160 \times 160$ & 298,236  \\
    \hline
    Maxpooling-2 & $64 \times 160 \times 160$ & $64 \times 80 \times 80$ &0  \\
    \hline
    DSWASPP-3  & $64 \times 80 \times 80$ & $128 \times 80 \times 80$ & 604,956  \\
    \hline
    Maxpooling-3 & $128 \times 80 \times 80$ & $128 \times 40 \times 40$ & 0  \\
    \hline
    DSWASPP-4  & $128 \times 80 \times 80$ & $128 \times 80 \times 80$ & 784,092 \\
    \hline
    DSWASPP-5  & $128 \times 80 \times 80$ & $128 \times 80 \times 80$ & 784,092 \\
    \hline
    DSWASPP-6  & $128 \times 80 \times 80$ & $128 \times 80 \times 80$ & 784,092 \\
    \hline
    SPP-1 & $128 \times 80 \times 80$ & $12800$ & 0 \\
    \hline
    Dropout-1 & $12800$ & $12800$ & 0 \\
    \hline
    FC-1 & $12800$ & $128$ & 1,638,528 \\
    \hline
    FC-2 & $128$ & $128$ & 16,512 \\
    \hline
    FC-3 & $128$ & $64$ & 8,256 \\
    \hline
    FC-4 & $64$ & $2$ & 130 \\
    \hline
    Softmax & $2$ & $2$ & 0 \\
    \hline
    \hline
    \multicolumn{3}{|c|}{Total Number of Parameter}&5,040,571\\
    \hline
    \multicolumn{4}{c}{Any linear combination is followed by BN and leakyReLU nonlinearity}\\
    \multicolumn{4}{l}{excluding re-projection layer of the SWASPP modules}
    \end{tabular}
    \label{PCNN}
    \end{center}
    \end{table}

\section{Summary}

CNN is a scale variant model. Atrous convolution is used to construct the scale space of the input feature. To select the correct scale of the input a spatial attention module is used.  A novel CNN architecture is proposed that internally produces multiscale feature maps. To learn a compact representation a bottleneck dimension is introduced in both the multiscale feature extractor module and the attention module.

 
% Chapter Template

\chapter{Experimental Results} % Main chapter title

\label{chp:results} % Change X to a consecutive number; for referencing this chapter elsewhere, use \ref{ChapterX}
In this chapter proposed methodologies are evaluated and compared with the related work.

All models are Trained using QaTa-Cov-19 \cite{ahishali2021advance} dataset using NVIDIA Tesla P-100 GPU and programmed using PyTorch.
\section{QaTa-COV19 Dataset}
\begin{center}
    \begin{figure}[htbp]
    \centerline{\includegraphics[height=60mm,width=9cm]{ScaleDist.png}}
    \caption{Pneumonia Scales of QaTa-COV19-v1, Y-axis represents the frequency, number of occurrence, of a pneumonia with a particular area}
    \label{pdist}
    \end{figure}
    \end{center}
QaTa-COV19 is a benchmark dataset for COVID19 detection and Segmentation form CXR images. All models that used for comparison are trained using QaTa-COV19-v1. Qata-COV19-v1 consist of  4603 COVID-19 CXR and  and $120,013$ control group CXRs. A balanced number of samples for the two classes is used, namely 4603 CXR image for each class to train the models. Pneumonia Scales of QaTa-COV19-v1 does not exhibit a uniform distribution. Scale of the Pneumonia can be defined as number, area, of 8-neighbor connected pixels labeled as COVID19 pneumonia. QaTa-COV19-v1 provides a binary masks of 2951 COVID19 CXR image which can be used for approximating the distribution of scales across the data set. Fig. \ref{pdist} illustrates the statistical distribution of QaTa-COV19 scales. The non-uniform distribution of the scales allows the CNN models to only recognize the small scales and not a large scales.

\section{Evaluation of the Methodology I}


Experiments are conducted on a Lenovo Z50-70 with Intel CORE i7-4510U CPU 2.00 GHz, 8GB RAM, NVIDIA GeForce 840M GPU; and with python and PyTorch library.

\subsection{Details of the Proposed Architecture}
The Proposed architecture composed Convbase and Densebase . Convbase is composed of a $6$ feature extraction modules \textit{(FX)} preceded by batch normalization layer as shown in Fig.5. Each FX module can be considered sub-sequential model consists of RSB layer followed by Batch Normalization, Max-pooling and LeakyReLU activation function. The Densebase is a two fully connected layers that classify the Convbase output.


\subsection{Hyperparameter Specification}
All input  chest X-Ray images are resized to be 200 $\times$ 200. After resizing the input images, these images are fed the Convbase model part which consists of 6 layers of residual separated block. Each residual separated block is followed with batch normalization and LeakyReLU \cite{he2015delving} as activation function  as shown in. The output depth of each  residual separated block is 4$\times$16, 4$\times$32, 4$\times$64, 4$\times$64, 4$\times$64 and 4$\times$16, respectively. The output of Convbase model part is 1D feature vector of 576 length. Densebase model part consists of two hidden layers. Each layer has the  size of 64 and the output layer of size 2. Each layer of Densebase layers is fully connected to its previous layer. The activation function used in the densebase model part is LeakyReLU. Table \ref{lyrSpec} summarizes the architecture hyperparameters.

\begin{table}[htbp]
\caption{The proposed architecture hyperparameters}
\begin{center}

\begin{tabular}{|l|c|c|}
\hline
\textbf{Layer Number} & \textbf{Layer Size} & \textbf{Activation Function} \\
\hline
\hline
RSBLayer1 & 4 $\times$ 16 & LeakyReLU\\
\hline
RSBLayer2 & 4 $\times$ 23 & LeakyReLU\\
\hline
RSBLayer3 & 4 $\times$ 64 & LeakyReLU\\
\hline
RSBLayer4 & 4 $\times$ 64 & LeakyReLU\\
\hline
RSBLayer5 & 4 $\times$ 64 & LeakyReLU\\
\hline
RSBLayer6 & 4 $\times$ 16 & LeakyReLU\\
\hline
\multicolumn{3}{|c|}{\textit{Flatten The Feature maps to 1D 576 feature  vector}}\\
\cline{1-3}
LinearLayer1 & 64 & LeakyReLU\\
\hline
LinearLayer2 & 64 & LeakyReLU\\
\hline
LinearLayer3 & 2 & Softmax\\
\hline
\end{tabular}
\label{lyrSpec}
\end{center}
\end{table}

\subsection{Network Training}
The proposed CNN model  is trained for 22 epoch. Adaptive Moment Estimation (Adam) optimizer \cite{kingma2014adam} is a popular optimization  technique for training deep networks. Adam optimizer is used  during the training  phase of the proposed CNN model. Both batch size and Adam optimizer learning rate is changed during the training phase if the training loss stopped decreasing. Table \ref{tabTrparam} summarizes the  parameters values used in the training phase of the proposed CNN model.  Fig. \ref{fig5}(a) show the progress for training and validation loss across each epoch. The difference between the training loss and validation loss through epochs show that our did not memorize the dataset.
\begin{table}[htbp]
\caption{The change of batch size and learning rate through the Training process}
\begin{center}

\begin{tabular}{|l|c|c|}
\hline
\textbf{Epoch Number} & \textbf{Batch Size} & \textbf{Learning Rate} \\
\hline
\hline
From 0 to 6 & 128 & 1e-3\\
\hline
From 7 to 12 & 256 & 1e-3\\
\hline
From 13 to 21 & 256 & 1e-4\\
\hline
 
\end{tabular}
\label{tabTrparam}
\end{center}
\end{table}



\subsection{Model Evaluation}

To assess the efficiency of the proposed method,  the proposed method is compared to recent state-of-the-art methods for detecting Covid-19 cases. Experiments are conducted with the same dataset and the corresponding hyperparameter of each work. All the methods depend on CNN. The comparison is performed using precision, sensitivity, F1-score, and accuracy \cite{hossin2015review}. In addition, the number of the parameters used in the training phase is  very important comparison factor. Table \ref{modelperf} depicts the comparison between state-of-the-art methods and the proposed method. As shown in the comparison, the proposed method  outperforms other methods achieving the maximum accuracy and the lowest  parameter count. 



\begin{table}[htbp]
\caption{ A performance comparison between the proposed method and state-of-the-art models.}
\begin{center}
\begin{tabular}{|l|c|c|c|c|c|}
\hline
\textbf{Method} & \textbf{PC} & \textbf{P(\%)}& \textbf{S(\%)}& \textbf{F1(\%)}& \textbf{A(\%)} \\
\hline
\hline
Proposed Method & 0.15M & 100.00 & 100.00 & 100.00 &100.00\\
\hline
ResNet-34 \cite{nayak2021application} & 21.8M & 96.77& 100.00 & 98.36 &98.33  \\
\hline
ACoS Phase I \cite{chandra2021coronavirus}& - & 98.266 & 96.512 & 98.551 & 98.062 \\
\hline
ResNet-50 \cite{nayak2021application}& 25.6M& 95.24& 100.00& 97.56& 97.50 \\
\hline
GoogleNet \cite{nayak2021application}& 5M &96.67& 96.67& 96.67& 96.67 \\
\hline
VGG-16 \cite{nayak2021application}& 138M& 95.08 & 96.67 & 95.87 &95.83\\
\hline
AlexNet \cite{nayak2021application}& 60M& 96.72 &98.33 & 97.52& 97.50 \\
\hline
MobileNet-V2 \cite{nayak2021application} & 3.4M &98.24& 93.33& 95.73 & 95.83 \\
\hline
Inception-V3 \cite{nayak2021application}& 24M &96.36& 88.33 & 92.17& 92.50\\
\hline
SqueezeNet \cite{nayak2021application}& 1.25M &98.27 &95.00& 96.61& 96.67 \\
\hline
\multicolumn{6}{l}{\textit{ PC is Parameter count, P is precision, S is sensitivity }}\\
\multicolumn{6}{l}{\textit{  F1 is F1-score, and A is accuracy }}\\
\hline
\end{tabular}
\label{modelperf}
\end{center}
\end{table}


\begin{figure}
\begin{center}
\includegraphics[height=90mm,width=8.0cm]{Figures/fig6.png}
\caption{(a) The training loss and the validation loss of each epoch and (b) The training accuracy and the validation accuracy of each epoch.}\label{fig5}\end{center}\end{figure}



\section{Evaluation of the Methodology II}
Methodology II is evaluated and compared against strong baselines and related works. 

\subsection{Baseline Networks}
Different architectures are trained to validate the effectiveness of the proposed method. 
\subsubsection{Spatial Pyramid Pooling (SPP-net) Based model}
A $4$ variants of SPP-net\cite{he2015spatial} is trained. All $4$ variant have the same architecture but different SPP-layer. These variants of SPP-layer are a full pyramid SPP of a 8-levels using average-pooling as aggregation function, full SPP pyramid of a 8-levels using max-pooling as aggregation function, single level SPP with 10-bins using average-pooling as aggregation function and single level SPP with 10-bins using max-pooling as aggregation function. A Fixed Architecture is used for all SPP variant models with the same design principles of the proposed architecture. This architecture is same as the proposed a architecture but DSWASPP is replaced by DC6 and SPP-1 layer is replaced with the corresponding SPP layer. DC6 is defined as six convolutional layer Densely connected together. For a SPP-net variants training a multiscale augmentation is added to the proposed augmentation process. multiscale augmentation is done by randomly sampling different $5$ scales typically $\{320, 320\pm25, 320\pm50 \}$.
\subsubsection{Switchable Atrous Spatial Pyramid Pooling (SASPP-net) Based models} 
\begin{center}
    \begin{figure*}[htbp]
    \centerline{\includegraphics[height=63mm,width=15cm]{SPP-netsTraining.PNG}}
    \caption{Training profiles of both SPP-net variants and the proposed network. For the same color solid line represent training statistics while dashed line represent the validation statistics for the corresponding model. \textbf{left:} is the training accuracy. \textbf{Right:} is the training loss.}
    \label{SPP-train}
    \end{figure*}
    \end{center}
Another Base-line is introduced for comparison which is exactly as same as the proposed network but with different Attention module structure and does not includes a bottleneck within ASPP. This architecture is referred Switchable Atrous Spatial Pyramid Pooling (SASPP-net). Attention module structure is $Softmax(FC(GAP(X)))$ where: \textit{$X$: is the input feature map}, \textit{$GAP$: is a global average pooling},  \textit{$FC$: is fully Connected layer performs a non-linear projection to ${\rm I\!R}^{4}$ a 4 values for the three scales and the input feature map}.\\

\begin{table}[htbp]
\caption{}
\begin{center}
\begin{tabular}{|c|c|c|}
\hline
\textbf{Model}&\multicolumn{2}{|c|}{\textbf{Baseline CNN Architectures}} \\
\cline{2-3} 
\textbf{Type} & \textbf{\textit{Variant}}& \textbf{\textit{Param. Count}} \\
\hline
  & ML Average pooling & $14,916,420$   \\
\cline{2-3} 
SPP & ML max pooling & $14,916,420$   \\
\cline{2-3} 
  & SL Average pooling & $14,490,436$   \\
\cline{2-3} 
  & SL max pooling & $14,490,436$ \\
\hline
\multicolumn{2}{|c|}{SASPP} & $13,031,841$\\
\hline
\multicolumn{3}{l}{ \textbf{ML}: Multilevel, \textbf{SL}: Single level}
\end{tabular}
\label{Basarch}
\end{center}
\end{table}
Table \ref{Basarch} summarizes the base-line models and the Corresponding parameter count.
\subsection{Models Training}
Proposed architecture and baseline architecture are trained with the same hyperparameters. Dataset is split to $0.6$, $0.2$ and $0.2$ for training, validation and testing, respectively. For training a Cross Entropy Loss is used. All models trained with ADAM \cite{kingma2014adam} optimizer with learning rate start by $10^{-3}$ and reduced every time validation loss plateau by multiplying by $10^{-1}$. A Max Norm Constraint is used to clip the gradient value to norm of $1$ \cite{krizhevsky2012imagenet}. A batch size of $128$ is used to calculate the gradient.
\subsection{Reducing the overfitting}
overfitting is a critical problem for training large networks \cite{krizhevsky2012imagenet}. Proposed work has reduced the overfitting by using:
\begin{itemize}
\item Using Dropout with a retrain probability of $0.5$ \cite{srivastava2014dropout}.
\item Using BatchNorm adds noise due to randomization introduced when constructing the minibatch \cite{ioffe2015batch}.
\item Using max norm constraint \cite{krizhevsky2012imagenet}.
\item deep and thin architectures by design has an implicit regularization effect \cite{he2016deep}. 
\item Augmentation process i.e.) Texture augmentation \cite{krizhevsky2012imagenet}. 
\item The use of small kernel size \cite{simonyan2014very}.
\item bottleneck in SWASPP module and the attention module.
\end{itemize}
During training no overfitting effects is observed.
\subsection{comparison with baselines}
Proposed network is compared with the vanilla SPP-based Architectures and ASPP architecture.
\subsubsection{comparing with SPP-nets}
Fig. \ref{SPP-train} illustrates both training loss and training and validation accuracies and losses. Table \ref{blaccom} illustrates the testing accuracy for comparison between the SPP-nets baseline and the proposed architecture.


\begin{table}[htbp]
\caption{Comparison between Proposed network and baseline SPP architectures }
\begin{center}
\begin{tabular}{|c|c|}
\hline
\textbf{Model Name}& Accurracy \\
\hline
 SPP ML Average pooling & $0.958$   \\
\hline
SPP ML max pooling & $0.950$   \\
\hline
  SPP SL Average pooling & $0.927$   \\
\hline
  SPP SL max pooling & $0.957$ \\
\hline
Proposed Network & $0.987$\\
\hline
\multicolumn{2}{l}{ \textbf{ML}: Multilevel, \textbf{SL}: Single level}
\end{tabular}
\label{blaccom}
\end{center}
\end{table}
\subsubsection{comparing with SASPP}
Fig. \ref{saspp} illustrates the training and validation loss of training a SASPP baseline architecture. As shown in Fig. \ref{saspp} SASPP unable to generalize and start overfitting the training set. This comparison empirically shows the importance of the bottleneck introduced in the proposed architecture.

\begin{center}
\begin{figure}[htbp]
\centerline{\includegraphics[height=55mm,width=8cm]{saspp.PNG}}
\caption{SASPP baseline architecture loss during both training, solid line, and validation, bashed line, compared with Proposed network and best performing SPP architecture.}
\label{saspp}
\end{figure}
\end{center}

\subsection{Comparing with the related works}
To fairly compare with the related works proposed work is further trained. Fig. \ref{ploss} shows the training and validation loss of the proposed network. Fig. \ref{pacc} shows the of the training and validation accuracy. 
\begin{center}
\begin{figure}[htbp]
\centerline{\includegraphics[height=55mm,width=8cm]{PLOSS.PNG}}
\caption{Cross entropy loss of the proposed architecture.}
\label{ploss}
\end{figure}
\end{center}
\begin{center}
\begin{figure}[htbp]
\centerline{\includegraphics[height=55mm,width=8cm]{PACC.PNG}}
\caption{Training and validation accuracy of the proposed architecture.}
\label{pacc}
\end{figure}
\end{center}
Proposed Network has a sensitivity, recall, and precision of $0.994$ and $0.991$ respectively on the validation set. precision can be improved by investigating the precision-recall trade-off. Fig. \ref{prt} shows the trade-off between precision and recall for different thresholds. A threshold of $0.618$ is used to improve the precision resulting in a sensitivity, recall, of $0.9903$ and precision of $0.9956$.
Comparison metrics are defined as follow:
\begin{itemize}
\item \textit{Accuracy}: is ratio of correctly classified samples to the total number of samples
\item \textit{Sensitivity}: is ratio of correctly classified Covid-19 samples to the total number of actual Covid-19 samples 
\item \textit{Precision}: is the ratio of correctly, according to the ground-truth labels, classified Covid-19 samples to the total number of samples classified as Covid-19.
\item \textit{Specificity}: is the ratio of correctly, according to the ground-truth labels, classified non-Covid19 to the total number of non-Covid-19.
\item \textit{F1-score}: is the harmonic mean of both Sensitivity and Precision.
\begin{center}  
 $F_{1}=\frac{2\times\text{Precision} \times \text{Sensitivity}}{\text{Precision} + \text{Sensitivity}}$
\end{center}
\item \textit{Param. Count}: is the total number of the trainable parameters.
\end{itemize}


\begin{table*}[!p!t]
\caption{Comparison between Proposed network and Related works }
\begin{center}
\begin{tabular}{|c|c|c|c|c|c|c|}
\hline
\textbf{Model Name}& \textbf{Accuracy} & \textbf{Sensitivity} &\textbf{ Precision} & \textbf{Specificity} & \textbf{F1-score} & \textbf{Param. Count}\\
\hline
\hline
Proposed & 0.99294 & \textbf{0.9903} & \textbf{0.9956} & 0.9956 & \textbf{0.9929} & \textbf{5,040,571}\\
\hline
SRC-Dalm\cite{ar} & 0.985 & 0.886 & - & 0.993 & - & -\\
\hline 
SRC-Hom\cite{ar} & 0.977 & 0.921 & - & 0.982 & - & - \\
\hline
CRC-light\cite{ar} & 0.973 & 0.955 & - & 0.974 & - &- \\
\hline
DenseNet121*\cite{ar} & 0.992 & 0.9714 & - & 0.9949 & - & 6,955,906  \\
\hline
Inception-v3\cite{ar} & 0.993 & 0.954 & - & 0.998 & - & 21,772,450  \\
\hline
Modified MobileNetV2 \cite{akt}  & 0.98 & 0.98 & 0.97 & - & 0.97 & -\\
\hline
ReCovNet-v2\cite{dag} & 0.99726 & 0.98571 & 0.94262 & 0.9977 & 0.96369 &- \\
\hline
ReCovNet-v1\cite{dag} & 0.99824 & 0.9781 & 0.97438 & 0.99901 & 0.97624 & -\\
\hline
DenseNet-121\cite{dag}  & \textbf{0.9988} & 0.97429 & 0.9932 & \textbf{0.99974} & 0.98365 & 6,955,906 \\
\hline
\end{tabular}
\end{center}
\label{rwcom}
\end{table*}


\begin{center}
\begin{figure}[htbp]
\centerline{\includegraphics[height=55mm,width=8cm]{PresRecuTradff.PNG}}
\caption{precision-recall trade-off of the proposed network.}
\label{prt}
\end{figure}
\end{center}


Table \ref{rwcom} summarizes the comparison between the recent related works and the proposed architecture. Proposed architecture outperform these works in many metrics. As their training and testing does not depend on a balanced number of samples, accuracy and specificity are not good metrics for  evaluation. 

% Chapter Template

\chapter{Conclusion} % Main chapter title

\label{chp:concl} % Change X to a consecutive number; for referencing this chapter elsewhere, use \ref{Chapter
% Chapter Template

\chapter{Summary} % Main chapter title

\label{chp:summ} % 

% COVID-19 is a severe respiratory tract infections. COVID-19 caused by SARS-CoV-2 can readily spread through a contact with an infected person. Monotonically increasing SARS-CoV-2 infections have not only wasted lives but also severely damaged the financial systems of both developing and developed countries. This high spread rate pressure on the health care systems which rise the need to fast methods for diagnosing this disease. Convolutional Neural Networks (CNN) show a great success for various computer vision tasks. However, CNN is a scale-variant model and computationally expensive. In this Thesis, a novel architectures are proposed for multiscale feature extraction and classification and lightweight architecture for COVID-19 diagnosing. The proposed I which is a lightweight CNN model exploits spatial kernel separability to reduce the number of the training parameters to a large extent and regularize the model to only learns linear kernel. Furthermore, This model uses residual connection and batch normalization extensively to maintain the network stability during the training process and provide the model with the regularization effect to reduce the overfitting. This lightweight architecture is trained using QaTa-Cov19 benchmark dataset achieving  100\% for accuracy, sensitivity, precision and F1-score with a very low parameter count (150K) compared with the other methods in the literature. As a future work attention and context attention can benefit the performance. Also evaluating atrous convolution can in the context of spatial separability can be beneficial. Proposed CNN II learns multiscale features using a pyramid of shared convolution kernels with different atrous rates. This scale invariant CNN uses attention based mechanism that is used to guide and select correct scale for each input. Proposed CNN II is an end-to-end trainable network and exploit a novel augmentation technique, Texture Augmentation, to reduce the overfitting. Proposed method II achieved a 0.9929 for $F1-score$ tested on QaTa-Cov19 benchmark dataset with a total of $5,040,571$ trainable parameters. SWASPP can be show a great performance for the segmentation specially atrous convolution originating at the segmentation literature, Also this work can be extended to classify a various pneumonia types.

The COVID-19 pandemic has caused severe respiratory tract infections that have rapidly spread through contact with infected individuals, resulting in devastating loss of life and economic damage worldwide. The high rate of transmission has put tremendous pressure on healthcare systems to develop fast and accurate methods for diagnosing the disease. Convolutional Neural Networks (CNNs) have shown success in various computer vision tasks, but they are scale-variant and computationally expensive. In this thesis, we proposed novel architectures for multiscale feature extraction and classification, as well as a lightweight architecture for COVID-19 diagnosis.

The proposed lightweight CNN model, referred to as CNN-I, exploits spatial kernel separability to significantly reduce the number of training parameters, and regularizes the model to only learn linear kernels. To maintain network stability and reduce overfitting, residual connections and batch normalization are extensively used. We trained this lightweight architecture on the QaTa-Cov19 benchmark dataset, achieving $100\%$ accuracy, sensitivity, precision, and F1-score with a parameter count of only 150K, which is significantly lower than other methods in the literature. As future work, attention and context attention can be explored to further enhance performance, and evaluating atrous convolution in the context of spatial separability may be beneficial.

Our second proposed architecture, CNN-II, learns multiscale features using a pyramid of shared convolution kernels with different atrous rates, making it scale-invariant. An attention-based mechanism is used to guide and select the correct scale for each input. CNN-II is an end-to-end trainable network that exploits a novel augmentation technique, Texture Augmentation, to reduce overfitting. This architecture achieved an F1-score of 0.9929 when tested on the QaTa-Cov19 benchmark dataset, with a total of 5,040,571 trainable parameters. We suggest that the SWASPP (Spatial Pyramid Atrous Spatial Pyramid Pooling) can show great performance for segmentation, especially atrous convolution originating in the segmentation literature. Additionally, this work can be extended to classify various types of pneumonia.

In conclusion, this thesis proposes novel architectures for COVID-19 diagnosis that address the limitations of traditional CNN models. These architectures achieved high accuracy while reducing computational cost and parameter count. Further research can explore attention mechanisms and evaluate the use of atrous convolution in the context of spatial separability to improve performance. This work has the potential to improve COVID-19 diagnosis and aid in the development of fast and effective methods to combat future pandemics.

\begin{itemize}
    \item \textbf{Chapter \ref{chp:intro}}: briefly discussed the history of COVID-19 and the importance of automating COVID-19 detection.
    \item \textbf{Chapter \ref{chp:background}}: includes required Background to understand the Thesis. 
    \item \textbf{Chapter \ref{chp:related}}: includes and illustrates the recent and related work in the COVID-19 detection literature. 
    \item \textbf{Chapter \ref{chp:proposed1}}: presents the proposed work I which presents a lightweight classification model.
    \item \textbf{Chapter \ref{chp:proposed2}}: presents the proposed work II which includes the scale invariant model for COVID-19 classification. 
    \item \textbf{Chapter \ref{chp:results}}: illustrates the experimental results for both proposed work I and II and quantitative analysis of the proposed work I and II is provided.
    \item \textbf{Chapter \ref{chp:concl}}: concludes the thesis and provide planning for the future work as extension of the proposed approach
\end{itemize}




%------------
%	THESIS CONTENT - APPENDICES
%---------------


% Include the appendices of the thesis as separate files from the Appendices folder
% Uncomment the lines as you write the Appendices

%\include{Appendices/AppendixA}
%\include{Appendices/AppendixB}
%\include{Appendices/AppendixC}

%---------------
%	BIBLIOGRAPHY
%---------------
% % Chapter Template

\chapter*{Publications} % Main chapter title
\addchaptertocentry{Publications}
\label{chp:publ} % 
\nocite{zaki2021covid} 
% \printbibliography[heading=none,keyword={Perhalo}]

\nocite{zaki2021covid} 
\nocite{zaki2022covid} 
\printbibheading[heading=bibintoc]
\printbibliography[keyword={publ}, heading=subbibliography, title={Publications}]
\printbibliography[title={References}]

\appendix % Cue to tell LaTeX that the following "chapters" are Appendices

\pagestyle{plain} % Default to the plain heading style until the thesis style is called for the body content
% Chapter Template

\chapter {\textarab[utf]{موجز الرسالة}} % Main chapter title
\label{araSummery} % Change X to a consecutive number; for referencing this chapter elsewhere, use \ref{ChapterX}
\begin{arab}[utf]

sfdsfdfsdf

\end{arab}
% Chapter Template
\begin{arab}[utf]
    % \appendix
    \chapter*{\textarab[utf]{نبذة الرسالة}} % Main chapter title
    \addchaptertocentry{\textarab[utf]{نبذة الرسالة}}
    \label{araSummery} 
    
        كوفيد19 هو التهاب حاد في الجهاز التنفسي. يمكن أن ينتشر كوفيد19 الناجم عن SARS-CoV-2 بسهولة من خلال الاتصال بشخص مصاب. إن التزايد الشديد للعدوى بفيروس SARS-CoV-2 لم يهدر الأرواح فحسب، بل أدى أيضًا إلى إلحاق أضرار جسيمة بالنظم المالية في كل من البلدان النامية والمتقدمة. هذا  المعدل العالى فى الانتشار يضع ضغط العالي على أنظمة الرعاية الصحية مما يزيد من الحاجة إلى طرق سريعة لتشخيص هذا المرض. تُظهر الشبكات العصبية التلافيفية (CNN) نجاحًا كبيرًا في مهام الرؤية الحاسوبية المختلفة. ومع ذلك ، يعد CNN نموذجًا متغيرًا scale-invariant ومكلفًا من الناحية الحسابية. في هذه الرسالة، تم اقتراح هياكل جديدة لاستخراج الميزات متعددة النطاقات وتصنيفها ونموزج خفيفة الوزن فى الحسابات لتشخيص كوفيد19. يستغل النموذج I المقترح وهو نموذج CNN خفيف الوزن فى الحسابات قابلية فصل kernel المكانية لتقليل عدد متغيرات التدريب إلى حد كبير وتنظيم النموذج لتعلم kernel خطية فقط. علاوة على ذلك ، يستخدم هذا النموذج الاتصال Residual BatchNorm وتطبيع الدُفعات على نطاق واسع للحفاظ على استقرار الشبكة أثناء عملية التدريب وتزويد النموذج بتأثير التنظيم Regularization لتقليل التجهيز الزائد Overfitting . يتعلم CNN II المقترح ميزات متعددة النطاقات Multiscale  باستخدام هرم Atrous Convolution المشتركة بمعدلات Dilation مختلفة. يستخدم  CNN  آلية قائمة على الانتباه Attention تُستخدم لتوجيه واختيار المقياس الصحيح لكل إدخال. إن شبكة CNN II المقترحة هي شبكة تدريب شاملة وتستغل تقنية زيادة جديدة ، وهي Texture Augmentation ، لتقليل التجهيز الزائد. يتم تدريب البنية خفيفة الوزن باستخدام مجموعة البيانات المعيارية QaTa-Cov19 التي تحقق 100 ٪ من الدقة والحساسية والدقة ودرجة F1 مع عدد معلمات منخفض جدًا (150K) مقارنة بالطرق الأخرى في الاعمال السابقه. حققت الطريقة المقترحة II 0.9929 لدرجة F1 التي تم اختبارها على مجموعة البيانات المعيارية QaTa-Cov19 بإجمالي 5،040،571 متغير قابلة للتدريب.
    
    \end{arab}
% Chapter Template

\chapter{\textarab[utf]{السيرة الذاتية لمقدم الرسالة} } % Main chapter title

\label{aracv} % Change X to a consecutive number; for referencing this chapter elsewhere, use \ref{ChapterX}
\begin{arab}[utf]

% \textarab[utf]{ العنوان }

\begin{itemize}
    \item \textbf{\textarab[utf]{ العنوان }}: \textarab[utf]{ سملا مركز }
\end{itemize}

\end{arab}

%------------

\end{document}  

% Chapter Template
\begin{arab}[utf]
% \appendix
\chapter*{\textarab[utf]{موجز الرسالة}} % Main chapter title
\addchaptertocentry{\textarab[utf]{موجز الرسالة}}
\label{araSummery} 
الجائحة الناجمة عن فيروس كورونا (COVID-19) تسببت في التهابات شديدة في المجاري التنفسية انتشرت بسرعة من خلال الاتصال بالأفراد المصابين، مما أدى إلى فقدان فادح للأرواح وتدمير اقتصادي على نطاق عالمي. وقد وضعت معدلات الانتقال العالية ضغطًا هائلًا على أنظمة الرعاية الصحية لتطوير طرق سريعة ودقيقة لتشخيص المرض. وقد أظهرت الشبكات العصبية التكرارية (CNNs) نجاحًا في مهام مختلفة في الرؤية الحاسوبية، ولكنها تعتبر متغيرة المقياس ومكلفة حسابيًا. في هذه الرسالة، اقترحنا تصميمات جديدة لاستخراج المعالم والتصنيف متعددة المقاييس، بالإضافة إلى تصميم خفيف الوزن لتشخيص COVID-19.

يستغل النموذج CNN-I الخفيف المقترح الفصلية المصفوفية للنواة الفضائية لتقليل عدد معاملات التدريب بشكل كبير، ويتم تنظيم النموذج ليتعلم فقط النوى الخطية. وتستخدم اتصالات البقاء والتعادل على مجموعات كبيرة للحفاظ على استقرار الشبكة وتقليل التكيف المفرط. قمنا بتدريب هذا التصميم الخفيف على مجموعة البيانات المرجعية QaTa-Cov19، وحققنا دقة وحساسية ودقة و F1-score بنسبة 100٪ بعدد معاملات تدريب قدره 150 ألف فقط، وهو أقل بكثير من الأساليب الأخرى في الأدبيات. وكعمل مستقبلي، يمكن استكشاف الاهتمام والاهتمام بالسياق لتعزيز الأداء بشكل أفضل، ويمكن أن يكون تقييم الانحراف الزمني في سياق الفصلية الفضائية مفيداً.

التصميم الثاني الذي قدمناه، CNN-II، يتعلم الميزات متعددة المقياسات باستخدام هرم من نوى التبعية المشتركة بمعدلات انحراف مختلفة، مما يجعله مقايسًا. يتم استخدام آلية قائمة على الاهتمام لتوجيه واختيار المقياس الصحيح لكل إدخال. CNN-II هو شبكة قابلة للتدريب من البداية إلى النهاية تستغل تقنية التعزيز الجديدة، التعزيز النسيجي، لتقليل التكيف المفرط. حقق هذا التصميم F1-score بنسبة 0.9929 عند اختباره على مجموعة بيانات المرجعية QaTa-Cov19، مع إجمالي 5،040،571 معاملًا قابلًا للتدريب. نقترح أن SWASPP (التجويف الهرمي الفضائي التجويف الهرمي الفضائي للتجميع) يمكن أن يظهر أداءً عظيمًا للتشريح، وخاصة الانحراف الزمني الناشئ في الأدبيات الخاصة بالتشريح. بالإضافة إلى ذلك، يمكن توسيع هذا العمل لتصنيف أنواع مختلفة من الالتهاب الرئوي.

في الخلاصة، تقدم هذه الرسالة العلمية تصاميم جديدة لتشخيص COVID-19 تتعامل مع القيود التي تواجه نماذج CNN التقليدية. حققت هذه التصاميم دقة عالية مع تقليل التكلفة الحسابية وعدد المعاملات. يمكن للأبحاث المستقبلية استكشاف آليات الانتباه وتقييم استخدام الانحراف الزمني الناشئ في السياق التبعي لتحسين الأداء. لدى هذا العمل القدرة على تحسين تشخيص COVID-19 والمساعدة في تطوير طرق سريعة وفعالة لمكافحة الأوبئة المستقبلية.

    وتقسم الرساله على النحو التالى:
    \begin{itemize}
        \item \textbf{\textarab[utf]{الفصل الاول}}: \textarab[utf]{يحتوى على مقدمة عن مرض كوفيد19 والمشكلة التي تبحثها الرسالة. أيضا يتم عرض الطريقة المقترحه بأختصار و الاسهامات التى تقدمها الرسالة فى هذا المجال.}
        \item \textbf{\textarab[utf]{الفصل الثاني}}: \textarab[utf]{ يعرض خلفية عامة عن الخوارزميات التي  سيتم استخدامها لاقتراح حل للتحسين التنبؤ مرض  كوفيد19.
هذا الفصل يعرض الطرق المقترحة سابقا والنتائج التي توصلوا اليها ومميزات وعيوب كل طريقة منهم.}
        \item \textbf{\textarab[utf]{الفصل الثالث}}: \textarab[utf]{يقدم وصفًا تفصيليًا عن اطار العمل المقترح الاول للتنبؤ بمرض كوفيد19 وأيضا كيف يتم تطبيقه على أحد البيانات الموجوده.
يقدم وصفًا تفصيليًا عن اطار العمل المقترح الثانى للتنبؤ بمرض كوفيد19.}
        \item \textbf{\textarab[utf]{الفصل الرابع}}: \textarab[utf]{يوضح ويناقش النتائج التي حققتها المقترحات.}
        \item \textbf{\textarab[utf]{الفصل الخامس}}: \textarab[utf]{يحتوي على عرض لما تم إنجازه في الرساله وكذلك بعض النقاط المقترحة لدراسات المستقبلية.}
        % \item \textbf{\textarab[utf]{}}: \textarab[utf]{}
    \end{itemize}
\end{arab}
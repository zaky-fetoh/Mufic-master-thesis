% Chapter Template
\begin{arab}[utf]
    % \appendix
    \chapter*{\textarab[utf]{نبذة الرسالة}} % Main chapter title
    \addchaptertocentry{\textarab[utf]{نبذة الرسالة}}
    \label{araSummery} 
    
        كوفيد19 هو التهاب حاد في الجهاز التنفسي. يمكن أن ينتشر كوفيد19 الناجم عن SARS-CoV-2 بسهولة من خلال الاتصال بشخص مصاب. إن التزايد الشديد للعدوى بفيروس SARS-CoV-2 لم يهدر الأرواح فحسب، بل أدى أيضًا إلى إلحاق أضرار جسيمة بالنظم المالية في كل من البلدان النامية والمتقدمة. هذا  المعدل العالى فى الانتشار يضع ضغط العالي على أنظمة الرعاية الصحية مما يزيد من الحاجة إلى طرق سريعة لتشخيص هذا المرض. تُظهر الشبكات العصبية التلافيفية (CNN) نجاحًا كبيرًا في مهام الرؤية الحاسوبية المختلفة. ومع ذلك ، يعد CNN نموذجًا متغيرًا scale-invariant ومكلفًا من الناحية الحسابية. في هذه الرسالة، تم اقتراح هياكل جديدة لاستخراج الميزات متعددة النطاقات وتصنيفها ونموزج خفيفة الوزن فى الحسابات لتشخيص كوفيد19. يستغل النموذج I المقترح وهو نموذج CNN خفيف الوزن فى الحسابات قابلية فصل kernel المكانية لتقليل عدد متغيرات التدريب إلى حد كبير وتنظيم النموذج لتعلم kernel خطية فقط. علاوة على ذلك ، يستخدم هذا النموذج الاتصال Residual BatchNorm وتطبيع الدُفعات على نطاق واسع للحفاظ على استقرار الشبكة أثناء عملية التدريب وتزويد النموذج بتأثير التنظيم Regularization لتقليل التجهيز الزائد Overfitting . يتعلم CNN II المقترح ميزات متعددة النطاقات Multiscale  باستخدام هرم Atrous Convolution المشتركة بمعدلات Dilation مختلفة. يستخدم  CNN  آلية قائمة على الانتباه Attention تُستخدم لتوجيه واختيار المقياس الصحيح لكل إدخال. إن شبكة CNN II المقترحة هي شبكة تدريب شاملة وتستغل تقنية زيادة جديدة ، وهي Texture Augmentation ، لتقليل التجهيز الزائد. يتم تدريب البنية خفيفة الوزن باستخدام مجموعة البيانات المعيارية QaTa-Cov19 التي تحقق 100 ٪ من الدقة والحساسية والدقة ودرجة F1 مع عدد معلمات منخفض جدًا (150K) مقارنة بالطرق الأخرى في الاعمال السابقه. حققت الطريقة المقترحة II 0.9929 لدرجة F1 التي تم اختبارها على مجموعة البيانات المعيارية QaTa-Cov19 بإجمالي 5،040،571 متغير قابلة للتدريب.
    
    \end{arab}
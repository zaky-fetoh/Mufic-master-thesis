% Chapter Template

\chapter{Conclusion And Future Work} % Main chapter title

\label{chp:concl} % 

% COVID-19 is a severe respiratory tract infection. COVID-19 caused by SARS-CoV-2 can readily spread through contact with an infected person. Monotonically increasing SARS-CoV-2 infections have not only wasted lives but also severely damaged the financial systems of both developing and developed countries. This high spread rate pressure on the health care systems which raises the need for fast methods for diagnosing this disease. Convolutional Neural Networks (CNN) show great success for various computer vision tasks. However, CNN is a scale-variant model and is computationally expensive. In this Thesis, novel architectures are proposed for multiscale feature extraction and classification and lightweight architecture for COVID-19 diagnosing. The proposed I which is a lightweight CNN model exploits spatial kernel separability to reduce the number of the training parameters to a large extent and regularize the model to only learn linear kernel. Furthermore, This model uses residual connection and batch normalization extensively to maintain the network stability during the training process and provide the model with the regularization effect to reduce overfitting. This lightweight architecture is trained using the QaTa-Cov19 benchmark dataset achieving  100\% for accuracy, sensitivity, precision, and F1-score with a very low parameter count (150K) compared with the other methods in the literature. As a future work attention and context attention can benefit the performance. Also evaluating atrous convolution in the context of spatial separability can be beneficial. Proposed CNN II learns multiscale features using a pyramid of shared convolution kernels with different atrous rates. This scale-invariant CNN uses an attention-based mechanism that is used to guide and select the correct scale for each input. Proposed CNN II is an end-to-end trainable network and exploits a novel augmentation technique, Texture Augmentation, to reduce overfitting. Proposed method II achieved a 0.9929 for $F1-score$ tested on the QaTa-Cov19 benchmark dataset with a total of $5,040,571$ trainable parameters. SWASPP can show a great performance for the segmentation especially atrous convolution originating in the segmentation literature, Also this work can be extended to classify various pneumonia types.

The COVID-19 pandemic has caused severe respiratory tract infections that have rapidly spread through contact with infected individuals, resulting in devastating loss of life and economic damage worldwide. The high rate of transmission has put tremendous pressure on healthcare systems to develop fast and accurate methods for diagnosing the disease. Convolutional Neural Networks (CNNs) have shown success in various computer vision tasks, but they are scale-variant and computationally expensive. In this thesis, we proposed novel architectures for multiscale feature extraction and classification, as well as a lightweight architecture for COVID-19 diagnosis.

The proposed lightweight CNN model, referred to as CNN-I, exploits spatial kernel separability to significantly reduce the number of training parameters and regularizes the model to only learn linear kernels. To maintain network stability and reduce overfitting, residual connections, and batch normalization are extensively used. We trained this lightweight architecture on the QaTa-Cov19 benchmark dataset, achieving $100\%$ accuracy, sensitivity, precision, and F1-score with a parameter count of only 150K, which is significantly lower than other methods in the literature. As future work, attention and context attention can be explored to further enhance performance, and evaluating atrous convolution in the context of spatial separability may be beneficial.

Our second proposed architecture, CNN-II, learns multiscale features using a pyramid of shared convolution kernels with different atrous rates, making it scale-invariant. An attention-based mechanism is used to guide and select the correct scale for each input. CNN-II is an end-to-end trainable network that exploits a novel augmentation technique, Texture Augmentation, to reduce overfitting. This architecture achieved an F1-score of 0.9929 when tested on the QaTa-Cov19 benchmark dataset, with a total of 5,040,571 trainable parameters. We suggest that the SWASPP (Spatial Pyramid Atrous Spatial Pyramid Pooling) can show great performance for segmentation, especially atrous convolution originating in the segmentation literature. Additionally, this work can be extended to classify various types of pneumonia.

In conclusion, this thesis proposes novel architectures for COVID-19 diagnosis that address the limitations of traditional CNN models. These architectures achieved high accuracy while reducing computational cost and parameter count. Further research can explore attention mechanisms and evaluate the use of atrous convolution in the context of spatial separability to improve performance. This work has the potential to improve COVID-19 diagnosis and aid in the development of fast and effective methods to combat future pandemics.
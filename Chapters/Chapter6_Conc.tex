% Chapter Template

\chapter{Conclusion And Future Work} % Main chapter title

\label{chp:concl} % 

COVID-19 is a severe respiratory tract infections. COVID-19 caused by SARS-CoV-2 can readily spread through a contact with an infected person. Monotonically increasing SARS-CoV-2 infections have not only wasted lives but also severely damaged the financial systems of both developing and developed countries. This high spread rate pressure on the health care systems which rise the need to fast methods for diagnosing this disease. Convolutional Neural Networks (CNN) show a great success for various computer vision tasks. However, CNN is a scale-variant model and computationally expensive. In this Thesis, a novel architectures are proposed for multiscale feature extraction and classification and lightweight architecture for COVID-19 diagnosing. The proposed I which is a lightweight CNN model exploits spatial kernel separability to reduce the number of the training parameters to a large extent and regularize the model to only learns linear kernel. Furthermore, This model uses residual connection and batch normalization extensively to maintain the network stability during the training process and provide the model with the regularization effect to reduce the overfitting. This lightweight architecture is trained using QaTa-Cov19 benchmark dataset achieving  100\% for accuracy, sensitivity, precision and F1-score with a very low parameter count (150K) compared with the other methods in the literature. As a future work attention and context attention can benefit the performance. Also evaluating atrous convolution can in the context of spatial separability can be beneficial. Proposed CNN II learns multiscale features using a pyramid of shared convolution kernels with different atrous rates. This scale invariant CNN uses attention based mechanism that is used to guide and select correct scale for each input. Proposed CNN II is an end-to-end trainable network and exploit a novel augmentation technique, Texture Augmentation, to reduce the overfitting. Proposed method II achieved a 0.9929 for $F1-score$ tested on QaTa-Cov19 benchmark dataset with a total of $5,040,571$ trainable parameters. SWASPP can be show a great performance for the segmentation specially atrous convolution originating at the segmentation literature, Also this work can be extended to classify a various pneumonia types.